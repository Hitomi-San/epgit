%! LuaLaTeX 文書
\usetheme{metropolis}
\usefonttheme{professionalfonts}

\hypersetup{linkcolor=,urlcolor=teal}

\usepackage{luatexja}
\ltjsetparameter{jacharrange={-2,-3,-8}}
\usepackage[no-math,match,deluxe]{luatexja-preset}

\usepackage{textcomp}
\usepackage{luatexja-otf}

\usepackage{graphicx,xcolor}
\usepackage{pxrubrica}
\usepackage{tcolorbox}
\usepackage{bxwareki}
\usepackage{indentfirst}

\usepackage{listings}
\lstset{
	doubleletterspace,
	basicstyle={\small\ttfamily},
	backgroundcolor={\color[gray]{.9}},
	commentstyle={\color[gray]{.5}},
	numbers=left,
	numberstyle=\tiny,
}

\usepackage{tikz}
\usetikzlibrary{plotmarks}

\usepackage{scsnowman} %☃
\usepackage{bxokumacro}
\usepackage{keystroke}
\usepackage{bxrawstr} % https://zrbabbler.hatenablog.com/entry/20181222/1545495849
\usepackage[twemoji-pdf]{bxcoloremoji} % https://github.com/zr-tex8r/BXcoloremoji

\usepackage{mflogo}

\usepackage{bxtexlogo}
\bxtexlogoimport{*,**}

\usepackage[T1]{fontenc}
\usepackage{mathtools,amssymb,mathrsfs,mathabx}
\usepackage[math]{iwona}
\usepackage[euler-digits]{eulervm}
\allowdisplaybreaks[4]

%%%%%%%%%% 商用フォントを用いているので、自分でタイプセットする場合は適当に以下を変えてください
\setmainfont[
	Ligatures=TeX,
	BoldFont=FOT-RodinNTLGPro-EB,
	ItalicFont=FOT-RodinNTLGPro-EB,
]{FOT-RodinNTLGPro-B}
\setsansfont[
	Ligatures=TeX,
	BoldFont=FOT-RodinNTLGPro-EB,
	ItalicFont=FOT-RodinNTLGPro-EB,
]{FOT-RodinNTLGPro-B}
\setmainjfont[
	Ligatures=TeX,
	CharacterWidth=Proportional,
	JFM=prop,
	BoldFont=FOT-RodinNTLGPro-EB,
	ItalicFont=FOT-RodinNTLGPro-EB,
]{FOT-RodinNTLGPro-B}
\setsansjfont[
	Ligatures=TeX,
	CharacterWidth=Proportional,
	JFM=prop,
	BoldFont=FOT-RodinNTLGPro-EB,
	ItalicFont=FOT-RodinNTLGPro-EB,
]{FOT-RodinNTLGPro-B}
\setmonofont[
	Ligatures=TeXReset,
]{HackGen}
\setmonojfont[
	Ligatures=TeXReset,
]{HackGen}

\newfontfamily\errorfont[
	Ligatures=TeX,
]{TsukuQMinLStd-L}

\newfontfamily\nishikifonta[
	Ligatures=TeX,
]{Nishiki-teki}
\newjfontfamily\nishikifontj[
	Ligatures=TeX,
]{Nishiki-teki}
\newcommand{\nishikifont}{\nishikifonta\nishikifontj}

%%%%%%%%%% 以下は自前コマンド

\newcommand{\centeralign}[1]{\rule{0pt}{0pt}\hfill#1\hfill\rule{0pt}{0pt}}
\newcommand{\typecommand}[1]{\colorbox{darkgray}{{\ttfamily\color{lime}#1}}}

\setlength{\parskip}{2ex}

\title{最低限 git 入門}
\author{北海道大学理学部 ひとみさん}
\date{\warekitoday}

\begin{document}

\frame{\maketitle}

\begin{frame}
	\frametitle{扱うことと扱わないこと}
	\begin{block}{扱うこと}
		\begin{itemize}
			\item git とは何か
			\item リポジトリの作り方
			\item コミットのしかた
		\end{itemize}
	\end{block}
	\begin{block}{扱わないこと}
		\begin{itemize}
			\item 複数人で開発する話
			\item ブランチ機能
		\end{itemize}
	\end{block}
\end{frame}

\begin{frame}
	\frametitle{参考文献}
	\begin{itemize}
		\item \textbf{Pro Git 2nd Edition}\\
			\url{https://git-scm.com/book/ja/v2}\\
			git 公式の日本語マニュアル
		\item \textbf{サルでもわかる Git 入門}\\
			\url{https://backlog.com/ja/git-tutorial/}\\
			とりあえずこれを読めば実用できるようになる
	\end{itemize}
\end{frame}

\begin{frame}
	\frametitle{git とはなにか}

	git とはバージョン管理システムである

	\begin{block}{バージョン管理システムとは}

		ファイルの変更履歴を記録する\\
		(記録するためのデータベース→\textbf{リポジトリ repository})

		\begin{itemize}
			\item いつ変更があったか参照できる
			\item 過去の状態に戻すことができる
		\end{itemize}
	\end{block}

	git はリポジトリをローカルとリモートに保存する
\end{frame}

\begin{frame}
	\frametitle{git を使うと有用な場合}
	\begin{itemize}
		\item \textbf{テキストデータを開発する場合}

			{\small 変更を参照しやすい(どの行を削除/追記したか)}
		\item \textbf{複数人で開発する場合}

			{\small リモートリポジトリから(他人が行った)変更を取り込める}

			{\small 複数人開発をしたことがないので、この話はないです}
		\item \textbf{開発に使う端末が複数ある場合}

			{\small リモートリポジトリ経由でデータを同期できる}\\
			{\small 本来の使い方ではなさそう(私はやっている)}
	\end{itemize}
\end{frame}

\begin{frame}
	\frametitle{準備}
	\begin{block}{git のインストール}
		適当なパッケージ管理システム経由でインストール

		\typecommand{\$ sudo apt install git}

		\typecommand{\$ git --version}\\
		\texttt{git version 2.24.0}
	\end{block}
	\begin{block}{設定}
		ユーザー名とメールアドレスを登録

		\typecommand{\$ git config -{}-global user.name "<ユーザ名>"}
		\typecommand{\$ git config -{}-global user.email "<メールアドレス>"}

		\texttt{\~{}/.gitconfig} に保存
	\end{block}
\end{frame}

\begin{frame}
	\frametitle{準備}
	\begin{block}{リモートリポジトリを保存する場所をつくる}
		GitHub\footnote{\url{https://github.com/}}にリモートリポジトリを置くのがメジャー

		GitHub にアカウントを作りましょう。

		世の中に GitHub しかないわけではないし、自前の git サーバーも作れる
	\end{block}
\end{frame}

\begin{frame}
	\frametitle{git の概要}
	リポジトリがリモートだけでなく、ローカルにもある\\
	→ほぼすべての作業をオフラインでできる\\
	→リモートサーバーが死んでも復元できる(分散管理)

	\begin{block}{3つの状態}
		\begin{figure}
			\centering
			\begin{tikzpicture}
				\coordinate(A1)at(0,0);
				\coordinate(A2)at(1,4);
				\coordinate(B1)at(3,0);
				\coordinate(B2)at(4,4);
				\coordinate(C1)at(6,0);
				\coordinate(C2)at(7,4);
				\coordinate(A3)at(1,3);
				\coordinate(A4)at(1,1);
				\coordinate(B3)at(3,3);
				\coordinate(B4)at(4,3);
				\coordinate(C3)at(6,1);
				\coordinate(C4)at(6,3);
				\draw(A1)rectangle(A2)node[midway]{\hbox{\tate 作業ディレクトリ}};
				\draw(B1)rectangle(B2)node[midway]{\hbox{\tate ステージングエリア}};
				\draw(C1)rectangle(C2)node[midway]{\hbox{\tate リポジトリ}};
				\draw[-stealth](C3)--(A4)node[near end,above]{\hbox{\scriptsize チェックアウト}};
				\draw[-stealth](A3)--(B3)node[near end,above]{\hbox{\scriptsize 変更・追加}};
				\draw[-stealth](B4)--(C4)node[near end,above]{\hbox{\scriptsize コミット}};
			\end{tikzpicture}
		\end{figure}
	\end{block}
\end{frame}

\begin{frame}
	\frametitle{git の利用}
	\begin{enumerate}
		\item \typecommand{git init}\\
			ローカルリポジトリの作成
		\item \typecommand{git remote add origin <url>}\\
			リモートリポジトリの追加
		\item \typecommand{git add .}\\
			新しいファイルの追跡、ステージングエリアに追加
		\item \typecommand{git commit -m "comment"}\\
			コミット
		\item \typecommand{git push origin master}\\
			リモートへプッシュ
	\end{enumerate}
\end{frame}

\begin{frame}
	\frametitle{git の利用}
	\begin{itemize}
		\item \typecommand{git pull}\\
			リモートからのプル
		\item \typecommand{git status}\\
			現在の状況の確認
		\item \typecommand{git rm}\\
			ファイルの削除(\typecommand{--cached}オプションで
			ステージングエリアからのみ除外)
		\item \textbf{.gitignore}\\
			ステージングエリアに自動で追加されない・変更を追跡されない
			ファイルを指定する。\\
			\url{https://github.com/github/gitignore}
	\end{itemize}
\end{frame}

\begin{frame}
	\frametitle{参考文献}
	\begin{itemize}
		\item \textbf{サルでもわかる Git 入門}\\
			\url{https://backlog.com/ja/git-tutorial/}\\
			とりあえずこれを読めば実用できるようになる
		\item \typecommand{\$ man git} \typecommand{\$ man gittutorial}\\
			「わからない時は \typecommand{man} を読め」
		\item \textbf{Pro Git 2nd Edition}\\
			\url{https://git-scm.com/book/ja/v2}\\
			git 公式の日本語マニュアル
	\end{itemize}
\end{frame}
\end{document}
